\pdfbookmark{Abstract}{Abstract}

\chapter*{Abstract}


\begin{tabular}{ | p{\dimexpr \linewidth-2\tabcolsep} |} \hline
 \begin{figure}[H]
        
        \includegraphics[width=0.2\textwidth]{images/indust} 
             \end{figure}  \\\hline
\end{tabular}   
\begin{tabular}{ | p{\dimexpr 0.3424\linewidth-2\tabcolsep} |
                  p{\dimexpr 0.3424\linewidth-2\tabcolsep} |
                  p{\dimexpr 0.3424\linewidth-2\tabcolsep} |} \hline
                 Approved & Examiner & Supervisor \\
                  \textbf{2015-MM-DD}  & \textbf{Martin Törngren} & \textbf{Baha Hasan} \\\hline
                   & Commissioner & Contact person \\
                    & \textbf{EXEGER Sweden AB} & \textbf{Camila Niva}\\ \hline
\end{tabular} \\

\begin{textblock}{8}(3,-3.3)
\begin{center}
\textbf{Master of Science Thesis MMK 2014: MF212X }
\end{center}
\end{textblock}
\begin{textblock}{8}[0.5,0.5](7,-2.3)
\begin{center}
\textbf{Optimized MPPT implementation for Dye-sensitized Solar cells}
\end{center}
\end{textblock}
\begin{textblock}{5}(8,-1.5)
Kartik Karuna
\end{textblock}

 European Commission's roadmap for moving to a low-carbon economy calls for a drastic reduction in the use of carbon based fuels. A low-carbon economy would have a much greater need for renewable sources of energy, energy-efficient building materials and other related technologies in order to reach  goals by 2050. More locally produced energy would be used, mostly from renewable sources with solar and wind playing an ever increasing role.Energy efficiency will be a key driver of this transition. Designing self-powered devices could offset a huge portion of the carbon budget, freeing devices from charging and making them truly wireless. \\
 
  Dye-Sensitized Solar Cells (DSCs) owing to their low-cost manufacturing technique among other advantages, could well be the missing piece in this puzzle. Lower efficiency vis-à-vis its Silicon based counterparts have dithered large scale implementation, however, with continued research that is soon to change. In order to  maximize the produced energy, several maximum power point tracking (MPPT) algorithms have been proposed and developed over the years. They vary in implementation, energy efficiency, convergence speed, sensors required, cost effectiveness etc. Although comparative studies, based on widely-adopted MPPT algorithms, have been presented before they focused on commercially available Silicon based Solar-cells; none have applied their findings to DSCs in indoor conditions. This work presents an experimental comparison, under simulated indoor irradiation, of three most used MPPT methods for PV power systems in an attempt to find one most suitable for DSCs. Subsequent experiments showed the existing MPPT methods to be unsuitable for DSCs leading to a new hybrid Algorithm being proposed.\\       

{\bf Keywords:} DSCs,Grätzel cells, MPPT, PnO, INC, Golden Search Algorithm, Machine Learning. 
\acresetall
