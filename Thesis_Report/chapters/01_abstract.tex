\pdfbookmark{Abstract}{Abstract}
\chapter*{Abstract}
\thispagestyle{begin}

% Here the abstract, remove the following command
(Filler text not actual text) During recent years, for photovoltaic (PV) systems, many maximum power point tracking (MPPT) algorithms have been proposed and developed to maximize the produced energy. Regarding the design manner, these methods vary in many aspects as: implementation simplicity, power or energy efficiency, convergence speed, sensors required, cost effectiveness. Some comparative studies, based on widely-adopted MPPT algorithms, presented in the literature give results obtained either from simulation tool, which provide simultaneous operating systems, or using real PV test bench under solar simulator in order to reproduce the same operating solar conditions. This work presents an experimental comparison, under real solar irradiation, of four most used MPPT methods for PV power systems: Perturb and Observe (P\&O) and Incremental Conductance, as tracking step constant, and improved P\&O and Fuzzy Logic based MPPT, as variable tracking step. Using four identical PV, under strictly the same set of technical and meteorological conditions, an experimental comparison of these four algorithms is done. Following two criteria, energy efficiency and cost effectiveness, this comparison shows the advantage of use of a MPPT with a variable tracking step. The extracted energies by all four methods are almost identical with a slight advantage for improved P\&O algorithm. \newline

**which can provide a convenient reference for future work in PV power generation, is based on the manner in which the control signal is generated and the PV power system behavior as it approaches steady state conditions**re write **

{\bf Keywords:} DSC, MPPT, PnO.
\acresetall
