\pdfbookmark{Sammanfattning}{Sammanfattning}
\chapter*{Sammanfattning}
\thispagestyle{begin}

\begin{tabular}{ | p{\dimexpr \linewidth-2\tabcolsep} |} \hline
 \begin{figure}[H]
        
        \includegraphics[width=0.2\textwidth]{images/indust} 
             \end{figure}  \\\hline
\end{tabular}   
\begin{tabular}{ | p{\dimexpr 0.3424\linewidth-2\tabcolsep} |
                  p{\dimexpr 0.3424\linewidth-2\tabcolsep} |
                  p{\dimexpr 0.3424\linewidth-2\tabcolsep} |} \hline
                 Godkänt & Examinator & Handledare \\
                  \textbf{2015-MM-DD}  & \textbf{Martin Törngren} & \textbf{Baha Hasan} \\\hline
                   & Uppdragsgivare & Kontaktperson \\
                   & \textbf{EXEGER Sweden AB} & \textbf{Camila Niva}\\ \hline
\end{tabular} \\
\begin{textblock}{8}(3,-3.3)
\begin{center}
\textbf{Examensarbete MMK 2014: MF212X }
\end{center}
\end{textblock}
\begin{textblock}{8}[0.5,0.5](7,-2.3)
\begin{center}
\textbf{Optimized MPPT implementation for Dye-sensitized Solar cells}
\end{center}
\end{textblock}
\begin{textblock}{5}(8,-1.5)
Kartik Karuna
\end{textblock}
% Here the abstract, remove the following command
Europeiska kommissionens plan för att skapa ett utsläppssnålt samhälle kräver en
drastisk minskning av användning av kolbaserade bränsle. Målet till 2050 är att bygga
ett ekonomiskt effektivt samhälle med förnybara energikällor, energieffektiva byggmaterial
och andra relaterade teknologier kommer att ge låga nivåer av koldioxidutsläpp. Lokalt producerad energi från förnybara källor kommer spela större roll i framtiden, mestadels från sol och vindenergi. Energieffektivitet kommer att drivkraften bakom denna övergång. Enheter som designas till att vara självförsörjande kommer att ge möjligheten till frigöring en stor del av kol budgeten, vilket kommer att frigöra enheter från att vara laddningsbara till att helt
gå över till att vara trådlösa.

En av fördelarna till att använda sig av färgsensibiliserade celler (dye-sensitized
cells, DSC) är dess låga tillverknings kostnader vilket kan vara den saknade pussel biten.
Lägre effektivitet gentemot sin kisel baserade motsvarighet är det största skälet varför ingen storskalig produktion finns, dock behövs det fortfarande göra mer forskning innan detta förändas. För att maximera den producerade energin, flera (maximum power point tracking,MPPT) algoritmer har utvecklats under åren. Algoritmernas implementation, energieffektivitet, konvergens hastighet, sensorer som krävs, kostnadseffektivitet etc varierar bland dessa algoritmer. Även om ett flertal studier baserats på allmänt
antagande om MPPT algoritm har presenterats innan, så har man man nu fokuserat
på kommersiellt tillgängliga Silicon baserade solar-celler. Ingen har tillämpat sina
iakttagelser på DSCs i inomhus förhållanden. Detta arbete presenterar en experimentell
jämförelse under simulerad inomhus bestrålning, med tre av de mest använda MPPT metoder
för PV kraftsystem i ett försök att hitta en som passar bäst för DSCs. Efterföljande
experiment visade de existerande MPPT metoder ej är lämpliga för DSCs vilket
leder till att en ny hybrid algoritm föreslås.\\
{\bf Nyckelord:} DSCs, Grätzel cells, MPPT, PnO, INC, Golden sökalgoritm, Maskininlärning. 
\acresetall
