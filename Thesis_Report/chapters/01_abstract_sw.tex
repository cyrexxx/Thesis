\pdfbookmark{Sammanfattning}{Sammanfattning}
\chapter*{Sammanfattning}
\thispagestyle{begin}

\begin{tabular}{ | p{\dimexpr \linewidth-2\tabcolsep} |} \hline
 \begin{figure}[H]
        
        \includegraphics[width=0.2\textwidth]{images/indust} 
             \end{figure}  \\\hline
\end{tabular}   
\begin{tabular}{ | p{\dimexpr 0.3424\linewidth-2\tabcolsep} |
                  p{\dimexpr 0.3424\linewidth-2\tabcolsep} |
                  p{\dimexpr 0.3424\linewidth-2\tabcolsep} |} \hline
                 Godkänt & Examinator & Handledare \\
                  \textbf{2015-MM-DD}  & \textbf{Martin Törngren} & \textbf{Baha Hasan} \\\hline
                   & Uppdragsgivare & Kontaktperson \\
                   & \textbf{EXEGER Sweden AB} & \textbf{Camila Niva}\\ \hline
\end{tabular} \\
\begin{textblock}{8}(3,-3.3)
\begin{center}
\textbf{Examensarbete MMK 2014: MF212X }
\end{center}
\end{textblock}
\begin{textblock}{8}[0.5,0.5](7,-2.3)
\begin{center}
\textbf{Optimized MPPT implementation for Dye-sensitized Solar cells}
\end{center}
\end{textblock}
\begin{textblock}{5}(8,-1.5)
Kartik Karuna
\end{textblock}
% Here the abstract, remove the following command
(**Temp text not actual translation **)
Europeiska kommissionens färdplan för ett utsläppssnålt samhälle kräver en drastisk minskning av användningen av kolbaserade bränslen. En ekonomi med låga koldioxidutsläpp skulle ha en mycket större behov av förnybara energikällor, energieffektiva byggmaterial och andra relaterade teknologier för att nå sina mål genom 2050. Mer lokalt producerad energi skulle användas, mestadels från förnybara källor med sol- och vinden spelar en allt större role.Energy effektivitet kommer att vara en viktig drivkraft för denna övergång. Designa själv-drivna enheter kan kompensera en stor del av kolet budgeten, vilket frigör enheter från att ta ut och göra dem riktigt trådlöst.\\

Färgsensibiliserade solceller eller Dye-sensitized solar cells (DSCs) på grund av deras låga kostnad tillverkningsteknik bland andra fördelar, kan mycket väl vara den saknade biten i detta pussel. Lägre effektivitet gentemot sin kiselbaserade motsvarigheter har dithered storskalig implementering, dock med fortsatt forskning som snart kommer att förändras. För att maximera den producerade energin, har flera spårning av maximal effektpunkt (maximum power point tracking,MPPT) algoritmer föreslagits och utvecklats under åren. De varierar i genomförandet, energieffektivitet, konvergens hastighet, sensorer krävs, kostnadseffektivitet etc. Även jämförande studier, baserade på allmänt antagna MPPT algoritmer har presente tidigare fokuserat på kommersiellt tillgängliga Silicon baserade Solar-celler; ingen har tillämpat sina iakttagelser till DSCs inomhus förhållanden. Detta arbete presenterar en experimentell jämförelse under simulerad inomhus bestrålning, av tre mest använda MPPT metoder för PV kraftsystem i ett försök att hitta en som passar bäst för DSCs. Efterföljande experiment visade de existerande MPPT metoder för att vara olämpliga för DSCs som leder till en ny hybrid algoritm föreslås.\\

{\bf Nyckelord:} DSCs, Grätzel cells, MPPT, PnO, INC, Golden sökalgoritm, Maskininlärning. 
\acresetall
