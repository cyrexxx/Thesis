\pdfbookmark{Sammanfattning}{Sammanfattning}
\chapter*{Sammanfattning}
\thispagestyle{begin}

\begin{tabular}{ | p{\dimexpr \linewidth-2\tabcolsep} |} \hline
 \begin{figure}[H]
        
        \includegraphics[width=0.2\textwidth]{images/indust} 
             \end{figure}  \\\hline
\end{tabular}   
\begin{tabular}{ | p{\dimexpr 0.3424\linewidth-2\tabcolsep} |
                  p{\dimexpr 0.3424\linewidth-2\tabcolsep} |
                  p{\dimexpr 0.3424\linewidth-2\tabcolsep} |} \hline
                 Godkänt & Examinator & Handledare \\
                  \textbf{2015-05-06}  & \textbf{Martin Törngren} & \textbf{Baha Hasan} \\\hline
                   & Uppdragsgivare & Kontaktperson \\
                   & \textbf{EXEGER Sweden AB} & \textbf{Camila Niva}\\ \hline
\end{tabular} \\
\begin{textblock}{8}(3,-3.3)
\begin{center}
\textbf{Examensarbete MMK 2015:15 MES 006 }
\end{center}
\end{textblock}
\begin{textblock}{8}[0.5,0.5](7,-2.3)
\begin{center}
\textbf{Optimized MPPT implementation for Dye-sensitized Solar cells}
\end{center}
\end{textblock}
\begin{textblock}{5}(8,-1.5)
Kartik Karuna
\end{textblock}
% Here the abstract, remove the following command
Europeiska kommissionens plan för att skapa ett utsläppssnålt samhälle kräver en drastisk minskning av användning av kolbaserade bränslen. Lokalt producerad energi från förnybara källor kommer spela större roll i framtiden, mestadels från sol och vindenergi. Energieffektivitet kommer vara drivkraften bakom denna övergång. Enheter som designas för att vara självförsörjande kommer att  bidra till en minskning av kolberoendet, frigöra enheter från att vara laddningsbara, och driva övergång till helt trådlösa enheter.\\

En av fördelarna till att använda sig av Grätzel-solceller (dye-sensitized cells, DSC)  är dess låga tillverkningskostnader. Lägre effektivitet gentemot kisel-baserade celler är det största skälet till att ingen storskalig produktion ännu finns, dock behövs det fortfarande mer forskning för att bana väg för större tillämpning. För att maximera den producerade energin,har flera (maximum power point tracking,MPPT) algoritmer har utvecklats under åren.  Algoritmerna har olika implementering, energieffektivitet, konvergenshastighet, sensorer som krävs, kostnadseffektivitet etc. Även om ett flertal studier presenterats, baserade på frekvent tillämpade MPPT algoritmer, så har dessa fokuserat på kommersiellt tillgängliga kisel-baserade solceller; inga studier har hittats där DSC-celler studerats för inomhus-tillämpningar. Detta arbete presenterar en experimentell jämförelse med simulerad inomhus-bestrålning, med tre av de mest använda MPPT metoderna för PV-kraftsystem i ett försök att hitta en som passar bäst för DSC’er. Efterföljande experiment visade de existerande MPPT metoder ej är lämpliga för DSCs. vilket föranledde utveckling och analys av en ny hybridalgoritm, som visar potential att snabbare (med färre iterationer) finna maximal effektpunkt.\\

{\bf Nyckelord:} DSCs, Grätzel cells, MPPT, PnO, INC, Golden sökalgoritm, Maskininlärning. 
\acresetall
