\pdfbookmark{Sammanfattning}{Sammanfattning}
\chapter*{Sammanfattning}
\thispagestyle{begin}

\begin{tabular}{ | p{\dimexpr \linewidth-2\tabcolsep} |} \hline
 \begin{figure}[H]
        
        \includegraphics[width=0.2\textwidth]{images/indust} 
             \end{figure}  \\\hline
\end{tabular}   
\begin{tabular}{ | p{\dimexpr 0.3424\linewidth-2\tabcolsep} |
                  p{\dimexpr 0.3424\linewidth-2\tabcolsep} |
                  p{\dimexpr 0.3424\linewidth-2\tabcolsep} |} \hline
                 Godkänt & Examinator & Handledare \\
                  \textbf{2015-MM-DD}  & \textbf{Martin Törngren} & \textbf{Baha Hasan} \\\hline
                   & Uppdragsgivare & Kontaktperson \\
                   & \textbf{EXEGER Sweden AB} & \textbf{Camila Niva}\\ \hline
\end{tabular} \\
\begin{textblock}{8}(3,-3.3)
\begin{center}
\textbf{Examensarbete MMK 2014: MF212X }
\end{center}
\end{textblock}
\begin{textblock}{8}[0.5,0.5](7,-2.3)
\begin{center}
\textbf{Optimized MPPT implementation for Dye-sensitized Solar cells}
\end{center}
\end{textblock}
\begin{textblock}{5}(8,-1.5)
Kartik Karuna
\end{textblock}
% Here the abstract, remove the following command
Europeiska kommissionens plan för att skapa ett utsläppssnålt samhälle kräver en drastisk minskning av användning av kolbaserade bränsle. Målet till 2050 är att bygga ett ekonomiskt effektivt samhälle med förnybara energikällor, energieffektiva byggmaterial och andra relaterade teknologier kommer att ge låga nivåer av koldioxidutsläpp. Där energin producerats lokalt från förnybarakällor, sol- vindenergi kommer att ha en större roll. Energieffektivitet kommer att drivkraften bakom denna övergång. Enheter som kommer att vara självdrivande kommer att bidra till att frigöra användning av kolet budget, vilket kommer att frigöra enheter från att vara laddningsbara till att helt gå över till att vara trädlösa. 

En av fördelarna till att använda sig  av färgsensibiliserade celler (dye-sensitized cells, DSC) är dess låga tillverknings kostnader som kan vara den saknade pussel biten. Lägre effektivitet gentemot sin kiselbaserade motsvarighet är utan tvekan den mest implementerade, dock behövs det fortfarande göra mer forskning. För att maximera den producerade energin, med flera spårning av maximal effektpunkters (maximum power point tracking,MPPT) algoritmer har utvecklats under åren. Algoritmernas implementation, energieffektivitet, konvergens hastighet, sensorer som krävs, kostnadseffektivitet etc variers bland dessa algoritmer. Även om ett flertal studier baserats på allmänt antagande om MPPT algoritm har presenterats innan, så har man man nu fokuserat på kommersiellt tillgängliga Silicon baserade solar-celler. Ingen har tillämpat sina iakttagelser till DSCs inomhus förhållanden. Detta arbete presenterar en experimentell jämförelse under simulerad inomhus bestrålning, av tre mest använda MPPT metoder för PV kraftsystem i ett försök att hitta en som passar bäst för DSCs. Efterföljande experiment visade de existerande MPPT metoder för att vara olämpliga för DSCs som leder till en ny hybrid algoritm som föreslås.\\
{\bf Nyckelord:} DSCs, Grätzel cells, MPPT, PnO, INC, Golden sökalgoritm, Maskininlärning. 
\acresetall
