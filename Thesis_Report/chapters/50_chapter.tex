\chapter{Conclusion and Future Work}
\begin{quote} 
\it This final chapter concludes the results obtained from the thesis and attempts to give direction for the future work in this area.
\end{quote}

While several \ac{MPPT} algorithms where studied in relation to this thesis , three of the most common ones were selected to be tested in this thesis. It was observed that each of the selected algorithms had one or more shortcoming when implemented on \ac{DSCs}    



% Commeted out.

%\subsection{**text**\cite{ngan2011study}}

%Ngan, Mei Shan, and Chee Wei Tan. "\textit{A study of maximum power point tracking algorithms for stand-alone photovoltaic systems.}" Applied Power Electronics Colloquium (IAPEC), 2011 IEEE. IEEE, 2011. \\

%\subsection{**text**\cite{esram2007comparison}}

%Esram, Trishan, and Patrick L. Chapman. "\textit{Comparison of photovoltaic array maximum power point tracking techniques.}" IEEE TRANSACTIONS ON ENERGY CONVERSION EC 22.2 (2007): 439.\\

%**almost all the Work in eltawil2013mppt is derived from this paper including pictures and text **
%\subsection{**text**\cite{eltawil2013mppt}}

%Eltawil, Mohamed A., and Zhengming Zhao. "\textit{MPPT techniques for photovoltaic applications.}" Renewable and Sustainable Energy Reviews 25 (2013): 793-813. \\

%This article aims to assess different MPPT techniques, provide background knowledge, implementation topology, grid interconnection of PV and solar microinverter requirements presented in the literature, doing in depth comparisons between them with a brief discussion. The MPPT merits, demerits and classification, which can be used as a reference for future research related to optimizing the solar power generation, are also discussed.\\

%\subsection{**text**\cite{reza2013classification}}

%Reza Reisi, Ali, Mohammad Hassan Moradi, and Shahriar Jamasb. "\textit{Classification and comparison of maximum power point tracking techniques for photovoltaic system: A review.}" Renewable and Sustainable Energy Reviews 19 (2013): 433-443.

%\subsection{**text**\cite{faranda2008energy}}

%Faranda, Roberto, and Sonia Leva. "\textit{Energy comparison of MPPT techniques for PV Systems.}" WSEAS transactions on power systems 3.6 (2008): 446-455.\\

%This paper presents a comparative study of ten widely-adopted MPPT algorithms; their performance is evaluated on the energy point of view, by using the simulation tool Simulink{\textregistered}, considering different solar irradiance variations.\\

%My thesis borrows some of the algorithms/Flow Chart mentioned in the above four papers. 



\subsection{**text**\cite{houssamo2013experimental}}

Houssamo, Issam, Fabrice Locment, and Manuela Sechilariu. "\textit{Experimental analysis of impact of MPPT methods on energy efficiency for photovoltaic power systems.}" International Journal of Electrical Power \& Energy Systems 46 (2013): 98-107.\\

This work presents an experimental comparison; Using four identical PV, under strictly the same set of technical and meteorological conditions, an experimental comparison  of four most used MPPT methods for PV power systems is done.This comparison shows the advantage of use of a MPPT with a variable tracking step.\\  

\subsection{**text**\cite{jain2004new}}
Jain, Sachin, and Vivek Agarwal. "\textit{A new algorithm for rapid tracking of approximate maximum power point in photovoltaic systems.}" Power Electronics Letters, IEEE 2.1 (2004): 16-19.\\

This paper presents a new algorithm for tracking maximum power point in photovoltaic systems. This is a fast tracking algorithm, where an initial approximation of \ac{MPP} quickly achieved using a variable step-size. Subsequently, the exact\ac{MPP} can be targeted using any conventional method like the hill-climbing or incremental conductance method. Thus, the drawback of a fixed small step-size over the entire tracking range is removed, resulting in reduced number of iterations and much faster tracking compared to conventional methods. \\

My implementation draws inspiration for the above article for its  two-stage algorithm to reduce the number of iterations but deviates significantly in the implementation and algorithms used to identify the \ac{MPP} 

\subsection{**text**\cite{liu2011fast}}

Liu, Yi-Hua, and Jia-Wei Huang. "\textit{A fast and low cost analog maximum power point tracking method for low power photovoltaic systems.}" Solar Energy 85.11 (2011): 2771-2780.\\

**add TEXT important RESEARCH * \\

Typically, MPPT methods utilized in medium and high power PV systems uses measured cell characteristics (current, voltage, power) along with an online search algorithm to compute the corresponding \ac{MPP}. Due to the complexity of the required mathematical operations, a \ac{DSP} or a relatively powerful micro-controller is typically needed, which increases the cost of the system.Moreover, it consumes significant portion of the generated power. Therefore, a\ac{MPPT} circuit with low-cost and fast-tracking features is essential .\\

It has already established that there exists a relation between V\textsubscript{MPP} and V\textsubscript{OC} in equation ~\ref{eq:equ_fracoc} and is famously used in the \ac{FOCV} method. However we see that this does not hold true for all illumination conditions and certainly not for low-light(less than ~1500 Lux ) as compared to higher insolation. Since V\textsubscript{OC} is is a logarithmic function of I\textsubscript{ph}, the relationship between  V\textsubscript{MP} and I\textsubscript{MP}.with respect to irradiation is not linear. However, it is possible to linearize this relationship for an interval where the value of V\textsubscript{OC} is sufficiently insensitive to irradiation. That is, the  \ac{VAL} can be calculated as the tangent line of the \ac{MPP} locus where the sensitivity of V\textsubscript{OC} to I\textsubscript{ph} is lower than a pre-defined threshold. This relationship is illustrated in Figure ~\ref{fig:Lui_IV_1} on page ~\pageref{fig:Lui_IV_1}.


 \begin{figure}[H]
  \begin{center}
  \includegraphics[width=\textwidth]{images/IVCurve_lui}
  \caption{I–V curves of the solar panel under different irradiation levels and the voltage approximation line. \cite{liu2011fast} }
  \label{fig:Lui_IV_1}
  \end{center}
  \end{figure}
  
  The same trend can be seen on the I-V curves for the \ac{DSCs} under test (Figure ~\ref{fig:vmmp_lux50_5000}).
   \begin{figure}[H]
    \begin{center}
    \includegraphics[width=\textwidth]{images/IV_50-500}
    \caption{Variation of V\textsubscript{MPP} for different illumination observed **in/on** the DSC under test }
    \label{fig:vmmp_lux50_5000}
    \end{center}
    \end{figure}
  

 \begin{figure}[H]
  \begin{center}
  \includegraphics[width=\textwidth]{images/IVCurve_lui_2}
  \caption{Research Cell Efficiency Records \cite{liu2011fast} }
  \label{fig:Lui_IV_2}
  \end{center}
  \end{figure}





**insert pictures**

\subsection{**text**\cite{dondi2008modeling}}

Dondi, Denis, et al. "\textit{Modeling and optimization of a solar energy harvester system for self-powered wireless sensor networks.}" Industrial Electronics, IEEE Transactions on 55.7 (2008): 2759-2766.\\


**Initial modelling strategies where based on this paper,  improving the efficiency. \# discrete components\# ** 

\subsection{**text**\cite{chu2009robust}}
Chu, Chen-Chi, and Chieh-Li Chen. "\textit{Robust maximum power point tracking method for photovoltaic cells: A sliding mode control approach.}" Solar Energy 83.8 (2009): 1370-1378.\\

\subsection{**text**\cite{urayai2011single}}
** Write about this in the proposed algo part**  
Urayai, Caston, and G. Amaratunga. "\textit{Single sensor boost converter-based maximum power point tracking algorithms.}" Applied Power Electronics Conference and Exposition (APEC), 2011 Twenty-Sixth Annual IEEE. IEEE, 2011.\\

Two new maximum power point tracking algorithms are presented: the input voltage sensor, and duty ratio maximum power point tracking algorithm (ViSD algorithm); and the output voltage sensor, and duty ratio maximum power point tracking algorithm (VoSD algorithm).unlike the incremental conductance algorithm which requires two sensors (the voltage sensor and current sensor), the two algorithms are more desirable because they require only one sensor: the voltage sensor.  \\ 

\subsection{**text**\cite{clark2006power}}

Clark, C., and A. Lopez. "\textit{Power system challenges for small satellite missions.}" Proceedings of the 2006 Small Satellites, Systems and Services Symposium, D. Danesy, Ed. The Netherlands: ESA. 2006.\\


**TExt**
Inspiration for use of multiple power buses 



