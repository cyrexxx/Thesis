\chapter{Conclusion and Future Work}
\begin{quote} 
\it This final chapter concludes the results obtained from the thesis and attempts to give direction for the future work in this area.
\end{quote}

Several \ac{MPPT} algorithms where studied in relation to this thesis ,However only three of the most common ones were selected to be tested in this thesis. It was observed that each of the selected algorithms had one or more shortcoming when implemented on \ac{DSCs}, This lead the author to propose his own algorithm best suited for the Cell at hand. The method shows promise and warrant further study. The author does not discount the fact that there might be several other algorithms in literature that might have an edge with respect to \ac{DSCs} and it could be worthwhile to invest resources to test them out.\\

Several assumptions were made with respect to the model that formed the basis for this research, further efforts could be put in to model the behaviour based on cell equations. Attempting to extract the unknown cell's parameters could merit a longer look. That being said I am of the impression that most \ac{MPPT} algorithms are essentially peak finding routines which imply that they would perform in a similar manner independent of the model or IV curves provided. Model based on the simplest diode equation may have indeed been sufficiently accurate within reasonable ranges,giving acceptable results for the purpose of this thesis.\\

Machine learning algorithms implemented are the most basic, smarter garbage collection in the buffer/look up table could improve the accuracy further. The process of formulating a new hybrid algorithm and testing this hypothesis out robbed valuable time that could have be utilised on developing test hardware. \ac{HIL} testing will definitely bring into focus new variables of energy efficiency, sensor count, Power budgeting of each measurement, cost of final hardware among others. \ac{HIL} simulations will also show if the assumptions made before still hold water. In essence \ac{HIL} testing is the logical next step for thesis work.\\

