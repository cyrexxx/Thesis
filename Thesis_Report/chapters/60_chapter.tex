\chapter{Conclusions And Future Work}

This study presents a base level attempt to find out how much of a user trajectory can be identified by an adversary with higher chance. Prospective non-safety applications in \ac{VANETS} were considered for emulation. These applications aim to provide users with certain
convenience and comfort applications. Though these applications promise to enhance user experience while driving, these also raise a concern towards users' privacy. Privacy analysis of non-safety applications is done and a base skeleton model was designed using heuristic methodology. The skeletal model aids in analysing different non-safety applications and their data exchange statistics between the vehicle and Infrastructure. We simulated realistic vehicular data with \ac{SUMO} simulator and the resultant dataset obtained was further used to emulate the use of non-safety applications. In addition, different adversarial setups and varying non-safety application running times were incorporated. Efforts were taken to replicate a realistic possible scenario.

After carrying out the planned research, results show that more the vehicles are exposed to non-safety application usage, the higher percentage of its trajectory is known to the adversary monitoring the communication channel. In addition, the results also show that an adversary with additional knowledge of crowded and dense road networks can determine significant user trip information. Further research into the topic can take advantage of all the results from the study.

The subjects for future work could include 1) Development of privacy enhancing approach that would provide a suitable balance between user privacy and quality of service for non-safety applications. 2) Real time simulation of vehicular movements, vehicular communication networks and application hosting/running can be carried upon and identifiers affecting the privacy of the user can be investigated. 3) Non-safety application privacy analysis skeletal model developed in this study can be updated with newer findings from the real time simulation as mentioned in point 2. 
